\chapter{Результаты}

\section{Анализ производительности предложенных алгоритмов}

Для сравнения производительности функций подсчета экспоненты и логарифма была взята реализация данных функций в рамках стандартной библиотеки языка OpenCL.
Для тестирования использовался закрытый драйвер OpenCL, поставляемый компанией Intel.
Тесты производились на машине под управлением операционной системы Linux Ubuntu.
Процессор машины Intel(R) Core(TM) i7-9700K CPU 3.60GHz.

\subsection{Тесты производительности}

В качестве теста производительности была предложена следующая программа.

Код устройства считал в цикле рассматриваемую функцию.
На каждой итерации функции подавалось одно и то же значение.
Чтобы оптимизации компилятора не сократили данный код до одной итерации, была введена искусственная зависимость аргумента функции от посчитанного значения функции на предыдущей итерации и входного аргумента вычислительного ядра.

Код для управляющего устройства как в случае теста на языке OpenCL, так и в случае случае теста на языке CM был написан с использованием API среды выполнения OpenCL.

И в случае OpenCL, и в случае CM вычислительное ядро запускалось на одном исполнительном устройстве, чтобы максимально исключить синхронизацию между потоками и работу среды исполнения по распределению задач из замеряемого времени.
Для того, чтобы получить эквивалентный код на двух языках, была определена степень векторизации выбранная компилятором IGC для OpenCL кода -- код был оттранслирован в режиме SIMD32, то есть одно исполнительное устройство обрабатывает вектор из 32 чисел.
Соответственно явные вектора из 32 элементов были использованы в коде, написанном на языке CM.

Тесты запускались с различными входными значениями.
Тесты на языке CM запускались в двух режимах: с оптимизацией алгоритма с помощью тэга и без.

\subsection{Сравнение производительности схемы Хорнера и метода Эстрина}

Сравнение схемы Хорнера и метода Эстрина было произведено на реализации экспоненты основанной на алгоритме из библиотеки glibc.
Замена метода Эстрина на схему Хорнера дала ускорение в 5\% на тесте производительности для экспоненты.
Входные значения были сформирован из чисел, дающих нормализованный результат, был использован тэг.

Преимущество схемы Хорнера над методом Эстрина для архитектуры \foreignlanguage{english}{Intel Processor Graphics} обусловлено тем, что в данной архитектуре не реализовано внеочередное исполнение команд.
То есть преимущество метода Эстрина, заключающееся в разбиении зависимостей по данным, не реализуется для данной архитектуры, а необходимые для данного алгоритма дополнительные вычисления замедляют алгоритм.

\subsection{Выбор алгоритма для подсчета экспоненты}

Алгоритм с непосредственным приближением экспоненты показал в 2.3 раза лучший результат по сравнению с алгоритмом из glibc реализации expm1.
Замер был произведен на входных данных, дающих нормализованный результат, была использована оптимизация тэгом.

Соответственно, в дальнейших замерах был использован алгоритм с непосредственным приближением экспоненты.

\subsection{Производительность функции подсчета экспоненты}

\begin{figure}[h]
  \centering
  \begin{tikzpicture}
  \begin{axis}[
    width = \textwidth,
    height = 0.5\textwidth,
    symbolic x coords={NUM,+UF,+OF,+NINF,+PINF,+NAN},
    xtick = data,
    xlabel = {Входные данные},
    xlabel near ticks,
    ylabel={Время исполнения, с},
    legend style={at={(0.5,-0.25)},
    anchor=north,legend columns=-1},
    ybar,
  ]
  \addplot
    coordinates {(NUM,4.732) (+UF,4.732) (+OF,4.752) (+NINF,4.966) (+PINF, 5.039) (+NAN, 5.16)};
  \addplot
    coordinates {(NUM,4.298) (+UF,4.298) (+OF,4.298) (+NINF,4.298) (+PINF, 4.3) (+NAN, 4.919)};
  \addplot
    coordinates {(NUM,4.913) (+UF,4.913) (+OF,4.913) (+NINF,4.913) (+PINF, 4.919) (+NAN, 4.919)};
  \legend{OpenCL,CM с тэгом,CM без тэга}
  \end{axis}
  \end{tikzpicture}
  \caption{Результат замеров производительности функции подсчета экспоненты}
  \label{hist:exp:perf}
\end{figure}

Для замера производительности число итераций цикла было выставлено в 4000000.

Было выделено 6 групп входных данных:

\begin{itemize}
    \item \textbf{NUM} --- случайные числа дающие при применении экспоненты результат, представимый нормализованным числом;
    \item \textbf{+UF} --- \textbf{NUM} плюс числа приводящие к переполнению снизу при вычислении экспоненты;
    \item \textbf{+OF} --- \textbf{+UF} плюс числа приводящие к переполнению сверху при вычислении экспоненты;
    \item \textbf{+NINF} --- \textbf{+OF} плюс отрицательная бесконечность;
    \item \textbf{+PINF} --- \textbf{+NINF} плюс положительная бесконечность;
    \item \textbf{+NAN} --- \textbf{+PINF} плюс NaN.
\end{itemize}

Результат замеров производительности функции подсчета экспоненты приведен на рисунке~\ref{hist:exp:perf}.


\subsection{Производительность функции подсчета логарифма}

\begin{figure}[h]
  \centering
  \begin{tikzpicture}
  \begin{axis}[
    width = \textwidth,
    height = 0.5\textwidth,
    symbolic x coords={POS,+DENORM,+ZERO,+NEG,+NINF,+PINF,+NAN},
    xtick = data,
    xlabel = {Входные данные},
    xlabel near ticks,
    ylabel={Время исполнения, с},
    legend style={at={(0.5,-0.25)},
    anchor=north,legend columns=-1},
    ybar,
  ]
  \addplot
    coordinates {(POS,2.983) (+DENORM,4.589) (+ZERO,4.735) (+NEG,4.752) (+NINF,4.850) (+PINF,4.859) (+NAN,4.879)};
  \addplot
    coordinates {(POS,4.604) (+DENORM,4.904) (+ZERO,5.078) (+NEG,5.091) (+NINF,5.236) (+PINF,5.292) (+NAN,5.506)};
  \addplot
    coordinates {(POS,5.370) (+DENORM,5.414) (+ZERO,5.423) (+NEG,5.444) (+NINF,5.482) (+PINF,5.506) (+NAN,5.506)};
  \legend{OpenCL,CM с тэгом,CM без тэга}
  \end{axis}
  \end{tikzpicture}
  \caption{Результат замеров производительности функции подсчета логарифма}
  \label{hist:log:perf}
\end{figure}

Для замера производительности число итераций цикла было выставлено в 2000000.

Было выделено 7 групп входных данных:

\begin{itemize}
    \item \textbf{POS} --- случайные положительные нормализованные числа;
    \item \textbf{+DENORM} --- случайные положительные числа: нормализованные и денормализованные;
    \item \textbf{+ZERO} --- \textbf{+DENORM} плюс ноль;
    \item \textbf{+NEG} --- \textbf{+ZERO} плюс отрицательные числа;
    \item \textbf{+NINF} --- \textbf{+NEG} плюс отрицательная бесконечность;
    \item \textbf{+PINF} --- \textbf{+NINF} плюс положительная бесконечность;
    \item \textbf{+NAN} --- \textbf{+PINF} плюс NaN.
\end{itemize}

Результат замеров производительности функции подсчета логарифма приведен на рисунке~\ref{hist:exp:perf}.

\section{Анализ точности предложенных алгоритмов}

\subsection{Экспонента}

Ошибка экспоненты не превышает 1 ULP на всей области определения.
График ошибки экспоненты на области $[-4\ln{2}, 4\ln{2}]$ представлен на рисунке~\ref{plot:exp:small}.

\begin{figure}[hbt]
  \centering
  \begin{tikzpicture}
  \begin{axis}[
    xtick = {-2.426, -1.733, -1.04, -0.347, 0, 0.347, 1.04, 1.733, 2.426},
    xticklabels = {$-\frac{7\ln{2}}{2}$, $-\frac{5\ln{2}}{2}$, $-\frac{3\ln{2}}{2}$, $-\frac{\ln{2}}{2}$, 0, $\frac{\ln{2}}{2}$, $\frac{3\ln{2}}{2}$, $\frac{5\ln{2}}{2}$, $\frac{7\ln{2}}{2}$},
    xlabel = {Аргумент функции},
    xlabel near ticks,
    ylabel = {Ошибка, ULP},
    ylabel near ticks,
    enlargelimits=false,
    width=\textwidth,
    height=0.6\textwidth]
  \addplot[
    only marks,
    scatter,
    mark=*,
    mark size=0.5pt,
    mark options={
      fill=black,
      draw=black,
    },
  ]
  table{data/exp_error_vast_dec.sample.dat};
  %full data set, too much for pgfplots
  %table{data/exp_error_vast_dec.dat};
  \end{axis}
  \end{tikzpicture}
  \caption{Ошибка предложенного алгоритма подсчета экспоненты на области $[-4\ln{2}, 4\ln{2}]$}
  \label{plot:exp:small}
\end{figure}

График ошибки экспоненты на всей области представимых значений \newline $[-744.5, 709.1]$ представлен на рисунке~\ref{plot:exp:whole}.

\begin{figure}[hbt]
    \centering
    \includegraphics[width=\textwidth]{exp_error_vast_whole.png}
    \caption{Ошибка предложенного алгоритма подсчета экспоненты на всей области представимых значений $[-744.5, 709.1]$}
    \label{plot:exp:whole}
\end{figure}

Слева на рисунке~\ref{plot:exp:whole} можно заметить малую область в которой ошибка не превышает $\frac{1}{2}$ ULP.
Это область денормализованного результата.
Относительная точность денормализованных чисел меньше чем у нормализованных.
Соответственно при получении денормализованного ответа из нормализованного, полученного на отрезке $[-\frac{\ln{2}}{2}, \frac{\ln{2}}{2}]$, значение правильно округлялось, правильное округление соответствует максимальной ошибке в $\frac{1}{2}$ ULP.

\subsection{Логарифм}

На области $[\frac{\sqrt{2}}{4}, 2\sqrt{2}]$ ошибка логарифма не превышает 2 ULP.
Превышение погрешности в 1 ULP обусловлено округлением при подсчете знаменателя в формуле~\ref{eq:log:var_impl}.

График ошибки логарифма на области $[\frac{\sqrt{2}}{8}, 4\sqrt{2}]$ представлен на рисунке~\ref{plot:log:small}.

\begin{figure}[hbt]
    \centering
    \includegraphics[width=\textwidth]{log_error_vast.png}
    \caption{Ошибка предложенного алгоритма подсчета логарифма на области $[\frac{\sqrt{2}}{8}, 4\sqrt{2}]$}
    \label{plot:log:small}
\end{figure}

Вне области $[\frac{\sqrt{2}}{4}, 2\sqrt{2}]$ ошибка логарифма не превышает 1 ULP.
График ошибки логарифма на всей области представимых значений $[2^{-1074}, 2^{1024})$ представлен на рисунке~\ref{plot:log:whole}.

\begin{figure}[hbt]
    \centering
    \includegraphics[width=\textwidth]{log_error_vast_whole.png}
    \caption{Ошибка предложенного алгоритма подсчета логарифма на всей области представимых значений $[2^{-1074}, 2^{1024})$}
    \label{plot:log:whole}
\end{figure}


\section{Выводы}

\subsection{Точность}

Точность предложенных решений сравнима с точностью реализации стандартной библиотеки OpenCL.

Реализация функции подсчета экспоненты в стандартной библиотеке языка OpenCL несколько точнее предложенного решения.
График ошибки OpenCL реализации экспоненты на области $[-4\ln{2}, 4\ln{2}]$ представлен на рисунке~\ref{plot:ocl_exp:small}.


\begin{figure}[hbt]
    \centering
    \includegraphics[width=\textwidth]{ocl_exp_error_vast.png}
    \caption{Ошибка OpenCL реализации экспоненты на области $[-4\ln{2}, 4\ln{2}]$}
    \label{plot:ocl_exp:small}
\end{figure}

Реализация функции подсчета логарифма в стандартной библиотеке языка OpenCL, как и в предложенном решении достигает максимальной ошибки порядка 2 ULP вблизи единицы.
При уменьшении или увеличении аргумента в два раза ошибка также, как в предложенном решении спадает до 1ULP.
График ошибки OpenCL реализации логарифма на области $[\frac{\sqrt{2}}{8}, 4\sqrt{2}]$ представлен на рисунке~\ref{plot:ocl_log:small}.


\begin{figure}[hbt]
    \centering
    \includegraphics[width=\textwidth]{ocl_log_error_vast.png}
    \caption{Ошибка OpenCL реализации логарифма на области $[\frac{\sqrt{2}}{8}, 4\sqrt{2}]$}
    \label{plot:ocl_log:small}
\end{figure}

Можно также заметить, что ошибка OpenCL реализации логарифма смещена относительно нуля на порядка 0.2 ULP: для чисел меньших единицы в большую сторону, для чисел больших единицы -- в меньшую.

\subsection{Производительность}

Производительность предложенных решений сравнима с производительностью решений стандартной библиотеки языка OpenCL.

Предложенная реализация экспоненты на рассматриваемом тесте без использования тэгов дала максимальный прирост по сравнению с реализацией OpenCL порядка 5\% для максимально разнородных входных данных, максимальный проигрыш составил примерно 4\% для максимально однородных данных.
Использование тэга ускорило алгоритм примерно на 15\%.

Предложенная реализация логарифма на рассматриваемом тесте проигрывает реализации стандартной библиотеки OpenCL.
Без использования тэгов проигрыш составляет от 12\% до 80\% (последние значение достигается только на одном наборе данных, если его исключить, проигрыш составит от 12\% до 17\%).
Использование тэга даёт ускорение от 4\% до 17\%.
Данный результат преимущественно обусловлен необходимость использовать дорогую инструкцию деления чисел с плавающей точкой для выбранного алгоритма.

\chapter{Заключение}

В рамках данной дипломной работы были рассмотрены существующие алгоритмы подсчета трансцендентных функций.
Изучены существующие реализации данных алгоритмов в рамках математических библиотеке.
На основе полученных данных были предложены реализации функций подсчета экспоненты и логарифма в рамках стандартной библиотеки языка CM.
Была произведена оптимизация данных решений для архитектуры \foreignlanguage{english}{Intel Processor Graphics}.
Был произведен замер точности и производительности данных решений.
Точность и производительность предложенных функции сравнимы с точностью и производительностью аналогичных решений для исследуемой платформы.
