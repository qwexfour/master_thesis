\chapter{Предложенный алгоритм для подсчета логарифма}

Предложенный алгоритм подсчета логарифма основан на интерпретации алгоритма Коди и Вейта в книге Нельсона Биба <<The Mathematical-Function Computation Handbook: Programming Using the MathCW Portable Software Library>>\cite{beebe}.

\section{Сокращение области}

Для сокращения области представим входной аргумент $arg$ в виде:

\begin{equation}
    arg = r \cdot 2^k
    \label{eq:log:reduce}
\end{equation}

где:

\begin{itemize}
    \item[] k --- целое.
\end{itemize}

В таком случае для результата можно получить:

\begin{equation}
    \ln{arg} = \ln{(r \cdot 2^k)} = \ln{r} + \ln{2^k} = \ln{r} + k \cdot \ln{2}
    \label{eq:log:reconstr}
\end{equation}

Для представления в виде~\ref{eq:log:reduce} возможно применить ранее реализованную функцию \textbf{frexp}.
При таком подходе область сократится до $r \in [0.5, 1)$.
Данная область не является оптимальной, так как вблизи единицы значение логарифма близко к нулю.
Таким образом, при области $[0.5, 1)$ даже если значения слева от единицы будут подсчитаны с точностью до правильно округленного значения логарифма, справа от единицы значения будут получены за счет значений полученных вблизи 0.5, абсолютная погрешность которых значительно больше необходимой абсолютной погрешности вблизи единицы, чтобы получить ошибку меньшую нескольких ULP.

Данную проблему можно избежать, если выбрать сокращенную область так, что единица находится в ее центре.
Этого результата можно добиться, если домножить элементы результата \textbf{frexp} на два в том случае, если они меньше $\frac{\sqrt{2}}{2}$.
Параметр \textit{k} в таком случае, соответственно, нужно сократить на единицу.
В результате данного преобразования рассматриваемая область станет равной $r \in [\frac{\sqrt{2}}{2}, \sqrt{2})$ с единицей в центре.

\section{Полиномиальное приближение}
\label{sec:work:log:approx}

Значение функции $\ln{r}$ при $r \in [\frac{\sqrt{2}}{2}, \sqrt{2})$ проходит через 0, из-за чего значение функции ULP на данном полуинтервале варьируется значительно.
Приближение функции $\ln{r}$ напрямую потребует слишком высокой степени полинома для получения ошибки в несколько ULP.

\subsection{Замена переменной}

Для того, чтобы обойти данную проблему, Коди и Вейтом\cite{cody-waite} была предложена замена переменной:

\begin{align}
    z(r) &= \frac{2(r - 1)}{r+1} \\
    r(z) &= \frac{2 + z}{2 - z}
\end{align}

Если $r \in [\frac{\sqrt{2}}{2}, \sqrt{2})$, то $z \in [-(6 - 4\sqrt{2}), 6 - 4\sqrt{2}) \approx [-0.343, 0.343)$.
Функция $z(r)$ монотонно растет на рассматриваемом интервале.

Для сокращения числа вычислений и увеличения точности, \textit{z} вычислялось по формуле:

\begin{equation}
    z = \frac{r - 1}{0.5\cdot r + 0.5}
    \label{eq:log:var_impl}
\end{equation}
В работе Нельсона Биба\cite{beebe} предлагалось вычислять $r - 1$, как $(r - 0.5) - 0.5$, для получения правильно округленного результата, однако данный подход не оправдан для арифметики, удовлетворяющей стандарту IEEE 754\cite{ieee754}, так как для математических операций гарантируется правильно округленный результат.
Эксперимент также подтвердил, что разбиение разности на две излишне.


При замене получаем функцию $f(z)$:

\begin{equation}
    f(z) = \ln{r(z)} = \ln{\frac{2 + z}{2 - z}}
\end{equation}

Легко заметить, что $f(z)$ нечетна:

\begin{equation}
    f(-z) = \ln{\frac{2 - z}{2 + z}} = -\ln{\frac{2 + z}{2 - z}} = -f(z)
\end{equation}

Учитывая нечетность функции, можно записать:

\begin{equation}
    f(z) = \ln{\frac{2 + z}{2 - z}} = z + z^3 \cdot R(z)
    \label{eq:log:f_to_R}
\end{equation}

Из \ref{eq:log:f_to_R} можно вывести:

\begin{equation}
    R(z) = \frac{\ln{\frac{2 + z}{2 - z}} - z}{z^3}
\end{equation}

\subsection{Приближение функции новой переменной}

Исходя из \ref{eq:log:f_to_R} и того, что функция $f(z)$ нечетная, можно заключить, что функция $R(z)$ четная.
На рассматриваемой области функция $R(z)$ принимает следующие значения $R(z) \in [\frac{1}{12}, 0.0848369\ldots] \subset (1.33 \cdot 2^{-4}, 1.36 \cdot 2^{-4})$.
Что означает, что $ULP(R(z))$ на рассматриваемой области принимает всего одно значение $2^{-52-4}=2^{-56}$.
Что в свою очередь делает возможным применение алгоритма Ремеза для приближения функции $R(z)$.

Подбором было установлено, что для приближения $R(z)$ с точностью выше 1 ULP, достаточного полинома 7-ой степени от $z^2$: $P_7(z^2)$.
Найденные значения коэффициентов:

\begin{align*}
    a_0 &= 8.333333333333333307726e-02 \\
    a_1 &= 1.250000000000027737181e-02 \\
    a_2 &= 2.232142857093574000534e-03 \\
    a_3 &= 4.340277811119014760447e-04 \\
    a_4 &= 8.877829843859590969703e-05 \\
    a_5 &= 1.878203829749719357775e-05 \\
    a_6 &= 4.049261716236508114464e-06 \\
    a_7 &= 9.985449030888993932992e-07
\end{align*}

\section{Восстановление значения на всей области}

Для восстановления значения логарифма на всей области применяется метод, схожий с методом сокращения области для экспоненты.
Данный метод также предложен Коди и Вейтом\cite{cody-waite}.

Для вычисления формулы~\ref{eq:log:reconstr} воспользуемся следующим выражением:

\begin{equation}
    \ln{arg} = (\ln{r} + k \cdot \ln{2}_{lo}) + k \cdot \ln{2}_{hi}
\end{equation}

Также как и в случае экспоненты $\ln{2}_{hi}$ и $\ln{2}_{lo}$ -- содержат старшие и младшие биты более точно представленной константы $\ln{2}$.
В качестве значений также использовались значения из листинга~\ref{lst:ln2_hi_lo}, так как в обоих случаях \textit{k} порядка $e_{max}$ или $e_{min}$.